\setcounter{chapter}{3}

\chapter{Derivadas parciais}

\resumo{titulo}{
\begin{itemize}[label=\color{chapterscolor}\textbullet]
    \item Entender o significado e a interpretação geométrica das derivadas parciais de funções de várias variáveis em relação a cada uma das variáveis independentes.

\item Os alunos devem ser capazes de calcular as derivadas parciais de funções de várias variáveis utilizando regras de diferenciação, como regra do produto, regra da cadeia e derivadas de funções elementares.
\end{itemize}
}


Vamos começar por lembrar a famosa lei da demanda. Basicamente essa lei nos diz que a quantidade demandada ($D$) de um bem diminui varia inversamente à medida que seu preço $p$ aumenta. Na linguagem matemática: a derivada da função de demanda em relação ao preço é negativa, ou seja, $D'(p)<0$. 

No entanto, em geral, a demanda por um produto não depende apenas do preço. 
Como vimos antes, a demanda por um bem ou serviço pode ser influenciada por diversos outros fatores, como a renda dos consumidores (\(R\)), os preços de bens substitutos (\(P_s\)) ou complementares (\(P_c\)) (entre outros). Considerando todos esses fatores, a função de demanda se torna mais complexa, refletindo a interdependência entre várias variáveis: \(D(p, R, P_s, P_c)\). 

Surge, então, uma pergunta crucial: como medir a sensibilidade da demanda a variações no preço? Em outras palavras, como podemos entender a taxa de variação da demanda quando o preço de um bem ou serviço muda? 

Neste momento podemos pensar que se consideramos $R$, $P_s$ e $P_c$ como sendo as constantes $r$, $p_s$ e $p_c$\footnote{Tais constantes devem ser escolhidas de modo que o ponto $(p,r,p_s,p_c)$ esteja dentro do domínio da função $D$.}, respectivamente, temos agora uma função que depende apenas do preço, e podemos derivar com respeito ao preço $p$. 

No entanto, ao considerar que essa função incorpora os outros fatores, percebermos que podemos e desejamos entender como a função muda em relação a cada um deles independentemente, entendemos que não podemos mais empregar a mesma notação familiar da Cálculo I, como \(D'\), uma vez que nesse contexto não ficaria claro em relação a qual dos fatores essa variação específica está relacionada. 

Surge então o conceito de derivada parcial. 
\begin{definition}{Derivada parcial}{def:der_parcial}
A \textit{derivada parcial}\index{derivada!parcial} de uma função \( f:D\subset\R^n\to\R \) em relação à variável \( x_i~(1\leq i \leq n) \) no ponto $\Point{x}=(x_1,x_2,\dots,x_n)\in D$ é denotada por \( \frac{\partial f}{\partial x_i} (\Point{x})\) e é definida como:
\[ \frac{\partial f}{\partial x_i} (\Point{x}) = \lim_{{h \to 0}} \frac{f(x_1,\dots,x_i + h,\dots x_n) - f(x_1,\dots,x_i,\dots x_n)}{h},\]
sempre que tal limite exista e for finito. 
\end{definition}

De forma intuitiva, o conceito de derivada parcial é uma extensão do conceito de derivada que considera uma função de várias variáveis. Enquanto a derivada de uma função de uma variável mede a taxa de variação instantânea ao longo de uma direção específica, a derivada parcial mede a taxa de variação instantânea de uma função em relação a uma variável específica, mantendo as outras variáveis constantes. 

Para exemplificar o conceito consideremos o caso de uma função de duas variáveis: $f:D\subset\R^2\to\R$. Se fixamos $y=k\in\R$, e consideramos os pontos $(x,k)\in D$, obtemos a curva
$$C=\{(x,k,f(x,k));~(x,k)\in D\}\subset \Gr(f),$$
que não é mais do que a interseção da superfície $\Gr(f)$ com o plano de equação $y=k$, ou seja, tal curva é uma curva plana paralela ao plano $xz$. Mais ainda, considerando a função $f_k(x)=f(x,k)$, vemos que $f_k'(x)=\frac{\partial f}{\partial x}(x,k)$. Concluímos, portanto, que  $\frac{\partial f}{\partial x}(h,k)$ é a inclinação da reta tangente à curva $C$ (vista como curva no plano de equação $y=k$) no ponto $(h,k)$. 



Analogamente, $\frac{\partial f}{\partial y}(h,k)$ é a inclinação da reta tangente à curva plana 
$$C=\{(h,y,f(h,y));~(h,y)\in D\}\subset \Gr(f),$$
(vista como curva no plano de equação $x=h$) no ponto $(h,k)$. 

\begin{example}{}{}
Consideremos a função $f(x,y)=\sqrt{1-x^2-y^2}$. Calculemos 
$$\frac{\partial f}{\partial x}\left(\frac{1}{2},-\frac{1}{2}\right)\quad \mbox{e}\quad \frac{\partial f}{\partial y}\left(\frac{1}{2},-\frac{1}{2}\right).$$ 
Primeiro, calculamos as funções 
$$\frac{\partial f}{\partial x}\left(x,y\right)\quad \mbox{e}\quad \frac{\partial f}{\partial y}\left(x,y\right).$$ 
Temos:
\begin{align*}
    \dfrac{\partial f}{\partial x}(x,y)&=\lim_{t\rightarrow 0}\dfrac{f(x+t,y)-f(x,y)}{t}.
%    &=\lim_{t\rightarrow 0}\dfrac{\sqrt{1-(x+h)^2-y^2}-\sqrt{1-x^2-y^2}}{t}.
\end{align*}
Como esse limite está indeterminado, podemos fazer transformações do argumento para tentarmos ``desfazer'' a indeterminação que, obviamente, depende de cada função. Para esta função faremos o seguinte desenvolvimento algébrico:
\begin{align*}
    &\frac{\partial f}{\partial x}(x,y)\\[.5em]
    =&\dfrac{\sqrt{1-(x+t)^2-y^2}-\sqrt{1-x^2-y^2}}{t}\\[.5em]
    =&\dfrac{\sqrt{1-(x+t)^2-y^2}-\sqrt{1-x^2-y^2}}{t}\cdot \dfrac{\sqrt{1-(x+t)^2-y^2}+\sqrt{1-x^2-y^2}}{\sqrt{1-(x+t)^2-y^2}+\sqrt{1-x^2-y^2}}\\
    =&\dfrac{-(2x+t)}{\sqrt{1-(x+t)^2-y^2}+\sqrt{1-x^2-y^2}}.
\end{align*}
Passando ao limite quando $t$ tende a zero obtemos
$$\dfrac{\partial f}{\partial x}(x,y)=-\dfrac{x}{\sqrt{1-x^2-y^2}}.$$
Analogamente,
$$\dfrac{\partial f}{\partial y}(x,y)=-\dfrac{y}{\sqrt{1-x^2-y^2}}.$$

Agora avaliamos essas duas funções no ponto desejado: 
$$\frac{\partial f}{\partial x}\left(\frac{1}{2},-\frac{1}{2}\right)=-\dfrac{\frac{1}{2}}{\sqrt{1-\left(\frac{1}{2}\right)^2-\left(-\frac{1}{2}\right)^2}}=-\dfrac{\sqrt{2}}{2}$$
e
$$\frac{\partial f}{\partial y}\left(\frac{1}{2},-\frac{1}{2}\right)=-\dfrac{-\frac{1}{2}}{\sqrt{1-\left(\frac{1}{2}\right)^2-\left(-\frac{1}{2}\right)^2}}=\dfrac{\sqrt{2}}{2}.$$

Na figura a seguir representamos o gráfico de $f$ (semiesfera superior), as curvas de intersecção entre o gráfico e os planos de equação $y=-\frac{1}{2}$ e $x=\frac{1}{2}$, respectivamente, ou seja, as curvas
$$\{x,-1/2,\sqrt{3/4-x^2});~-\sqrt{3}/2\leq x\leq  \sqrt{3}/2\}$$
e
$$\{1/2,y,\sqrt{3/4-y^2});~-\sqrt{3}/2\leq y\leq  \sqrt{3}/2\},$$
e também as retas tangentes cuja inclinação nos planos $xz$ e $yz$ são, respectivamente, 
$$\frac{\partial f}{\partial x}\left(\frac{1}{2},-\frac{1}{2}\right)$$
e
$$\frac{\partial f}{\partial y}\left(\frac{1}{2},-\frac{1}{2}\right).$$
\begin{figure}[H]
    \centering
\tdplotsetmaincoords{70}{110} % Adjust the view angles (azimuth, elevation)



\begin{tikzpicture}[tdplot_main_coords,scale=1.5]
%parte de trás do círculo vermelho
\draw[red!140,very thick] plot[domain=0:sqrt(3)/2,smooth,variable=\t] ({-\t},{-0.5},{sqrt(3/4-\t^2)});
		\pgfmathsetmacro{\tini}{0.5*pi}
		\pgfmathsetmacro{\tfin}{1.85*pi}
		\pgfmathsetmacro{\tend}{2.5*pi}
		% Node indicating the equation of the circumference
%		\draw[white] (1.35,0,0) -- (0,1.35,0) node [red,below,midway,sloped] {$x^2 + y^2 = 1$};
		%%% Coordinate axis
		\draw[-latex] (0,0,0) -- (1.5,0,0) node [below left] {$x$};
		\draw[dashed] (0,0,0) -- (-1.25,0,0);
		\draw[-latex] (0,0,0) -- (0,1.5,0) node [right] {$y$};
		\draw[dashed] (0,0,0) -- (0,-1.25,0);
		% The region of integration
		%\fill[yellow,opacity=0.35] plot[domain=0:6.2832,smooth,variable=\t] ({cos(\t r)},{sin(\t r)},{0.0});
		%\draw[red,thick] plot[domain=0:6.2832,smooth,variable=\t] ({cos(\t r)},{sin(\t r)},{0.0});
		% The curves slicing the surface
	%\draw[blue!140] plot[domain=-1:1,smooth,variable=\t] ({\t},0,{\t*\t}); 
		%\draw[blue!140,opacity=0.5] plot[domain=-1:1,smooth,variable=\t] (0,{\t},{\t*\t}); 
		% El paraboloid (for z = constant)
		\foreach \altura in {0.0125,0.025,...,1.0}{
			\pgfmathparse{sqrt(1-\altura^2)}
			\pgfmathsetmacro{\radio}{\pgfmathresult}
			\draw[blue!50,opacity=0.5] plot[domain=\tini:\tfin,smooth,variable=\t] ({\radio*cos(\t r)},{\radio*sin(\t r)},{\altura}); 
		}

  
		% Circunference bounding the surface (above, first part)
%		\draw[blue,opacity=0.75] plot[domain=pi:1.75*pi,smooth,variable=\t] ({cos(\t r)},{sin(\t r)},{1.0}); 
		% last part of the z axis
		\draw[-latex] (0,0,0) -- (0,0,1.75) node [above] {$z$};	
		\foreach \altura in {0.0125,0.025,...,1.0}{
			\pgfmathparse{sqrt(1-\altura^2)}
			\pgfmathsetmacro{\radio}{\pgfmathresult}
			\draw[blue!50,opacity=0.5] plot[domain=\tfin:\tend,smooth,variable=\t] ({\radio*cos(\t r)},{\radio*sin(\t r)},{\altura}); 
		}
		% Circunference bounding the surface (above, last part)
%		\draw[blue,opacity=0.75] plot[domain=-0.25*pi:pi,smooth,variable=\t] ({cos(\t r)},{sin(\t r)},{1.0}); 

    
\draw[dashed,gray!50] (0.5,-0.5,0) -- (0.5,-0.5,{2*(0.5)^2});
\draw[dashed,gray!50] (0.5,0,0) -- (0.5,-0.5,0)--(0,-0.5,0);



\draw[red!140,very thick,opacity=0.5] plot[domain=0:sqrt(3)/2,smooth,variable=\t] ({\t},{-0.5},{sqrt(3/4-\t^2)});


\draw[red!140,very thick,opacity=0.5] plot[domain=-1:1,smooth,variable=\t] ({1/2+\t},{-1/2},{sqrt(2)/2-\t*sqrt(2)/2});

%\draw[yellow!140,very thick,opacity=0.5] plot[domain=-sqrt(3)/2:sqrt(3)/2,smooth,variable=\t] ({0.5},{\t},{sqrt(3/4-(\t)^2)});

%\draw[yellow!140,very thick,opacity=0.5] plot[domain=-1:1,smooth,variable=\t] ({1/2},{-1/2-\t},{sqrt(2)/2-\t*sqrt(2)/2});

\filldraw[red!150] (0.5,-0.5,{sqrt(1-2*(0.5)^2)}) circle (1pt) node[left]{\footnotesize$\left(\frac{1}{2},-\frac{1}{2}\right)$};

%\node[above right] at (-1,0.5,0.75) {$x_3 = x_1^2 + x_2^2$};
  
	\end{tikzpicture}
\hfill
\tdplotsetmaincoords{70}{110} % Adjust the view angles (azimuth, 
\begin{tikzpicture}[tdplot_main_coords,scale=1.5]
		\pgfmathsetmacro{\tini}{0.5*pi}
		\pgfmathsetmacro{\tfin}{1.85*pi}
		\pgfmathsetmacro{\tend}{2.5*pi}
		% Node indicating the equation of the circumference
%		\draw[white] (1.35,0,0) -- (0,1.35,0) node [red,below,midway,sloped] {$x^2 + y^2 = 1$};
		%%% Coordinate axis
		\draw[-latex] (0,0,0) -- (1.5,0,0) node [below left] {$x$};
		\draw[dashed] (0,0,0) -- (-1.25,0,0);
		\draw[-latex] (0,0,0) -- (0,1.5,0) node [right] {$y$};
		\draw[dashed] (0,0,0) -- (0,-1.25,0);
		% The region of integration
		%\fill[yellow,opacity=0.35] plot[domain=0:6.2832,smooth,variable=\t] ({cos(\t r)},{sin(\t r)},{0.0});
		%\draw[red,thick] plot[domain=0:6.2832,smooth,variable=\t] ({cos(\t r)},{sin(\t r)},{0.0});
		% The curves slicing the surface
	%\draw[blue!140] plot[domain=-1:1,smooth,variable=\t] ({\t},0,{\t*\t}); 
		%\draw[blue!140,opacity=0.5] plot[domain=-1:1,smooth,variable=\t] (0,{\t},{\t*\t}); 
		% El paraboloid (for z = constant)
		\foreach \altura in {0.0125,0.025,...,1.0}{
			\pgfmathparse{sqrt(1-\altura^2)}
			\pgfmathsetmacro{\radio}{\pgfmathresult}
			\draw[blue!50,opacity=0.5] plot[domain=\tini:\tfin,smooth,variable=\t] ({\radio*cos(\t r)},{\radio*sin(\t r)},{\altura}); 
		}

  
  
		% Circunference bounding the surface (above, first part)
%		\draw[blue,opacity=0.75] plot[domain=pi:1.75*pi,smooth,variable=\t] ({cos(\t r)},{sin(\t r)},{1.0}); 
		% last part of the z axis
		\draw[-latex] (0,0,0) -- (0,0,1.75) node [above] {$z$};	
		\foreach \altura in {0.0125,0.025,...,1.0}{
			\pgfmathparse{sqrt(1-\altura^2)}
			\pgfmathsetmacro{\radio}{\pgfmathresult}
			\draw[blue!50,opacity=0.5] plot[domain=\tfin:\tend,smooth,variable=\t] ({\radio*cos(\t r)},{\radio*sin(\t r)},{\altura}); 
		}
		% Circunference bounding the surface (above, last part)
%		\draw[blue,opacity=0.75] plot[domain=-0.25*pi:pi,smooth,variable=\t] ({cos(\t r)},{sin(\t r)},{1.0}); 

    
\draw[dashed,gray!50] (0.5,-0.5,0) -- (0.5,-0.5,{2*(0.5)^2});
\draw[dashed,gray!50] (0.5,0,0) -- (0.5,-0.5,0)--(0,-0.5,0);



%\draw[orange,very thick,opacity=0.5] plot[domain=0:sqrt(3)/2,smooth,variable=\t] ({\t},{-0.5},{sqrt(3/4-\t^2)});

%\draw[orange,very thick,opacity=0.5] plot[domain=-1:1,smooth,variable=\t] ({1/2+\t},{-1/2},{sqrt(2)/2-\t*sqrt(2)/2});

\draw[red!140,very thick,opacity=0.5] plot[domain=-sqrt(3)/2:sqrt(3)/2,smooth,variable=\t] ({0.5},{\t},{sqrt(3/4-(\t)^2)});


\draw[red!140,very thick,opacity=0.5] plot[domain=-1:1,smooth,variable=\t] ({1/2},{-1/2-\t},{sqrt(2)/2-\t*sqrt(2)/2});

\filldraw[red!150] (0.5,-0.5,{sqrt(1-2*(0.5)^2)}) circle (1pt) node[left]{\footnotesize$\left(\frac{1}{2},-\frac{1}{2}\right)$};

%\node[above right] at (-1,0.5,0.75) {$x_3 = x_1^2 + x_2^2$};
  
	\end{tikzpicture}
    \caption{}
%    \label{fig:enter-label}
\end{figure}

\end{example}


É evidente que calcular o limite a cada vez que desejamos obter uma derivada parcial não é prático. Ao diferenciar em relação a uma variável, mantendo as outras constantes, fica claro que podemos empregar as mesmas regras e técnicas de diferenciação utilizadas para calcular derivadas de funções de uma variável.

\begin{proposition}{Regras de Diferenciação}{prop:regras_der}
Sejam $f$ e $g$ funções \red{diferenciáveis} em $D\subset\R^n$ e $c\in\R$. Temos:
\begin{itemize}[label=\color{resultscolor!150}\textbullet]
\item\textbf{Regra da Soma}:
   \[
   \frac{\partial (f+g)}{\partial x_i}(\Point{x}) = \frac{\partial f}{\partial x_i}(\Point{x}) + \frac{\partial g}{\partial x_i}(\Point{x}).
   \]


 

\item \textbf{Regra do Produto}:
   \[
   \frac{\partial (f\cdot g)}{\partial x_i}(\Point{x}) = f (\Point{x})\cdot \frac{\partial g}{\partial x_i}(\Point{x})+ g (\Point{x})\cdot\frac{\partial f}{\partial x_i} (\Point{x}) .
   \]

\item \textbf{Regra da Constante}:
\[
   \frac{\partial (c\cdot f)}{\partial x_i}(\Point{x}) = c \cdot\frac{\partial f}{\partial x_i} (\Point{x}) .
   \]

\item \textbf{Regra da cadeia I}: Se $f(\Point{x})=g(u_1(\Point{x}),u_2(\Point{x}),\dots,u_n(\Point{x}))$, onde para cada $1\leq i \leq n$ existe $\frac{\partial u_i}{\partial x_i}(\Point{x})$, então
$$\frac{\partial f}{x_i}(\Point{x})=\sum_{j=1}^n \frac{\partial g}{\partial u_j}(u_1(\Point{x}),u_2(\Point{x}),\dots,u_n(\Point{x}))\cdot \frac{\partial u_j}{\partial x_i}.  $$

   \end{itemize}
\end{proposition}

\begin{example}{}{}
Calculemos as derivadas parciais do exemplo anterior usando as regras de derivação mencionadas.     
\end{example}




\begin{theorem}{Teorema da Função Implícita}{}
Seja \(g: \mathbb{R}^n \to \mathbb{R}\) uma função definida em uma vizinhança \(D\) de um ponto \(\Point{a}=(a_1, a_2, \ldots, a_n) \in \mathbb{R}^n\), e suponha que \(g\) seja continuamente diferenciável em \(D\). Se \(g(\Point{a}) = 0\) \(\frac{\partial g}{\partial x_n}(\Point{a})\neq 0\), então existe uma vizinhança \(\hat{D} \subset D\) de \(\Point{a}\) e uma única função \(f: \mathbb{R}^{n-1} \to \mathbb{R}\) que é \textbf{continuamente diferenciável} em \(\hat{D}\), tal que para cada \(\hat{\Point{x}}=(x_1, x_2, \ldots, x_{n-1}) \in \hat{D}\), temos \(g(\hat{\Point{x}}, f(\hat{\Point{x}})) = 0\).
\end{theorem}



A \textit{grosso modo}, sob certas condições, podemos encontrar uma função implícita que descreve uma das variáveis de \(f\) em relação às outras \(n-1\) variáveis, mesmo que a equação não possa ser resolvida explicitamente. Isso é particularmente útil quando tratamos de sistemas de equações complicados onde não é possível isolar facilmente uma das variáveis. O teorema nos fornece um meio de entender como as variáveis estão inter-relacionadas em tais situações.


\begin{example}{}{}
Consideremos a esfera de equação $x^2+y^2+z^2=1$. Queremos aplicar o teorema anterior. 
Seja $g(x,y,z)=x^2+y^2+z^2-1$. Então $g(0,0,1)=0$ pois $(0,0,1)$ é o polo norte da esfera. Além disso, $\frac{\partial g}{\partial z}(0,0,1)=2\neq 0$. Logo, existe uma vizinhança de  $\hat{D}$ de $(0,0)$ em $\R^2$ e uma função $f:\hat{D} \to \R$ tal que $z=f(x,y)$. 
Neste exemplo \textit{já sabemos} que  $f(x,y)=\sqrt{1-x^2-y^2}$ e $\hat{D}$ é o círculo bidimensional centrado na origem e de raio 1. 
\end{example}

O exemplo da esfera é bastante concreto, o que nos permite identificar claramente a função \(f\) e o domínio \(\hat{D}\). Entretanto, em situações gerais, essa clareza pode não estar presente, muitas vezes devido a equações complexas que podem ser difíceis de resolver, mesmo em nível local. No entanto, na prática, muitas vezes não precisamos conhecer a expressão exata de \(f\), mas sim entender o comportamento da variável $x_n=f(\hat{\Point{x}})$ em relação às mudanças das outras. É exatamente nesse contexto que ganha importância o teorema da função implícita. 

\begin{corollary}{}{}
    Sob as hipóteses do teorema, temos, para $1\leq i\leq n-1$,  
    $$\frac{\partial x_n}{\partial x_i}=-\dfrac{\frac{\partial g}{\partial x_i}}{\frac{\partial g}{\partial x_n}}.$$
\end{corollary}
\begin{proof}
    Basta aplicar a regra da cadeia à equação 
    \[g(\hat{\Point{x}}, f(\hat{\Point{x}})) = g(x_1,\dots,x_{n-1},f(x_1,\dots,x_{n-1}))=0.\] 
\end{proof}





\section{Derivada direcional}

Sabemos que a derivada parcial de uma função $f:\R^n\to\R$ é o seguinte limite se existir:
\begin{equation}\label{der_parcial}
    \frac{\partial f}{\partial x_i} (\Point{x}) = \lim_{{h \to 0}} \frac{f(x_1,\dots,x_i + h,\dots x_n) - f(x_1,\dots,x_i,\dots x_n)}{h},
\end{equation} 
Vamos olhar a variação na coordenada $x_i$ desde um outro ponto de vista: 
\begin{align*}
(x_1,\dots,x_i + h,\dots, x_n)&=(x_1,\dots,x_i ,\dots, x_n)+(0,\dots,\underbrace{h}_{\textup{posição~}i},\dots,0)\\
&=(x_1,\dots,x_i ,\dots x_n)+h(0,\dots,\underbrace{1}_{\textup{posição~}i},\dots,0)\\
&=\Point{x}+h\cdot \Vec{e_i},
\end{align*}
onde $\Point{x}=(x_1,\dots, x_n)$ e $\Vec{e_i}$ é o $i$-ésimo vetor canônico de $\R^n$. %Sendo $h\in\R$ muito pequeno --- pois estamos calculando um limite quando $h\rightarrow 0$, 
Vemos, então, que nos estamos acercando a $\Point{x}$ ao longo de um pequeno seguimento cujo vetor diretor é $\Vec{e_i}$. 
Então podemos reescrever \eqref{der_parcial} como
\begin{equation}\label{der_parcial}
\frac{\partial f}{\partial x_i} (\Point{x}) = \lim_{{h \to 0}} \frac{f(\Point{x}+h\cdot \Vec{e_i}) - f(\Point{x})}{h}.
\end{equation} 
Ou seja, estamos querendo saber a variação da função  ao longo de um pequeno seguimento paralelo aos  eixos coordenados. 

Então, faz sentido pensar no seguinte limite 
\begin{equation}\label{der_parcial}
%\frac{\partial f}{\partial x_i} (\Point{x}) = 
\lim_{{h \to 0}} \frac{f(\Point{x}+h\cdot \Vec{v}) - f(\Point{x})}{h},
\end{equation} 
onde $\Vec{v}$ é um vetor qualquer de $\R^n$.

\begin{definition}{Derivada direcional}{def:der_direcional}
A \textit{derivada direcional}\index{derivada!parcial} de uma função \( f:D\subset\R^n\to\R \) na direção de um vetor $\Vec{v}\in\R^n$  no ponto $\Point{x}=(x_1,x_2,\dots,x_n)\in D$ é denotada por \( \frac{\partial f}{\partial \Vec{v}} (\Point{x})\) e é definida como:
\[ \frac{\partial f}{\partial \Vec{v}} (\Point{x}) = \lim_{{h \to 0}} \frac{f(\Point{x}+h\cdot \Vec{v}) - f(\Point{x})}{h},\]
sempre que tal limite exista e for finito. 
\end{definition}

\begin{example}{}{}
    $$\frac{\partial f}{\partial x_i}(\Point{x})=\frac{\partial f}{\partial \Vec{e_i}}(\Point{x}).$$
\end{example}

\begin{theorem}{Fórmula da derivada direcional}{}
    Se \( f:D\subset\R^n\to\R \) é diferenciável em $\Point{x}$, então 
    $\frac{\partial f}{\partial \Vec{v}} (\Point{x})$ existe para qualquer $\Vec{v}\in\R^n$ e 
    $$\frac{\partial f}{\partial \Vec{v}} (\Point{x})=\sum_{i=1}^nv_i\frac{\partial f}{\partial x_i}(\Point{x}),$$
    onde $v_1,\dots,v_n$ são as coordenadas do vetor $\vec{v}$. 
\end{theorem}

Observe que se definimos o seguinte vetor:
$$\left(\frac{\partial f}{\partial x_1}(\Point{x}), \dots, \frac{\partial f}{\partial x_n}(\Point{x})  \right),$$
então temos que 
$$\frac{\partial f}{\partial \Vec{v}} (\Point{x})=\escalar{(v_1,\dots,v_n)}{\left(\frac{\partial f}{\partial x_1}(\Point{x}), \dots, \frac{\partial f}{\partial x_n}(\Point{x})  \right)}.$$
Tal vetor é de grande importância e o definimos a seguir. 

\begin{definition}{Vetor gradiente}{}
Seja  \( f:D\subset\R^n\to\R \) uma função diferenciável em $\Point{x}$. Chamamos de \textit{vetor gradiente} de $f$ em $x$ o vetor definido por
    $$\grad f(\Point{x})=\left(\frac{\partial f}{\partial x_1}(\Point{x}), \dots, \frac{\partial f}{\partial x_n}(\Point{x})  \right).$$
\end{definition}
Temos, então,
$$\frac{\partial f}{\partial \Vec{v}} (\Point{x})=\escalar{\grad f(\Point{x})}{\vec{v}}.$$

Observe agora que pela fórmula do produto escalar
$$%\frac{\partial f}{\partial \Vec{v}} (\Point{x})=
\escalar{\grad f(\Point{x})}{\vec{v}}=\|\Vec{v}\| \| \grad f(\Point{x})\| \cos \theta, $$
onde $\theta$ é o ângulo entre os vetores $\vec{v}$
 e $\grad f(\Point{x})$. Sabemos que o valor máximo da função $\cos$ é $1$, atingido quando $\theta=0$, logo, $\frac{\partial f}{\partial \Vec{v}}(\Point{x})$ é máxima quando $\vec{v}$ tem a mesma direção e sentido que $\grad f(\Point{x})$. Concluímos, portanto, que a direção de maior crescimento de uma função a partir de um ponto $\Point{x}$ é $\grad f(\Point{x})$. 
 