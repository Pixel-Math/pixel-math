\section{Dominio?????????????????}

Observe agora que, para que a função \(D\) do exemplo introdutório faça sentido na vida real, é importante que cada uma das quatro variáveis independentes seja não negativa. Em outras palavras, as variáveis \(p\), \(R\), \(P_s\) e \(P_c\) devem satisfazer \(p \geq 0\), \(R \geq 0\), \(P_s \geq 0\) e \(P_c \geq 0\) para garantir que o preço do produto, a renda dos consumidores e os preços dos bens substitutos e complementares estejam dentro de um contexto viável economicamente. 

Isso nos induz a pensar que nem sempre devemos considerar todas as combinações do espaço \(\mathbb{R}^4\), mas sim um subconjunto dele. 


Para que a função \(D\) seja significativa e bem-definida na vida real, é necessário que as variáveis independentes \(p\), \(R\), \(P_s\) e \(P_c\) estejam dentro de um subconjunto apropriado do espaço \(\mathbb{R}^4\). Esse subconjunto pode ser determinado pelas condições econômicas e práticas do problema em questão.

Por exemplo, é comum restringir as variáveis independentes a valores não negativos, como mencionado anteriormente, pois preços e renda negativos não fazem sentido no contexto econômico. Além disso, outras restrições podem ser impostas, como limites superiores para os preços ou renda, com base em limitações do mercado ou regulamentações governamentais.

Portanto, ao analisar a função \(D\) em um cenário econômico específico, é essencial considerar cuidadosamente o conjunto apropriado no espaço \(\mathbb{R}^4\) que reflete a realidade do problema. 

