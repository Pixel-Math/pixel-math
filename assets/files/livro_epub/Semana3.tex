%%%%%%%%%%%%%%%%%%%%%%%%
\setcounter{chapter}{2}

\chapter{Limite e continuidade}

\resumo{titulo}{
\begin{itemize}[label=\color{chapterscolor}\textbullet]
    \item Entender a noção de limite de uma função de várias variáveis quando nos aproximamos de um ponto específico no domínio.
    
    \item Compreender a definição formal de limite em termos de $\varepsilon$-$\delta$.
    
    \item Explorar as propriedades dos limites, incluindo as regras de soma, produto, constante e quociente.
    
    \item Compreender o conceito de limite ao longo de caminhos e como eles são relevantes na definição de limites de funções de várias variáveis.
    
    \item Entender a definição de continuidade de uma função $f(x, y)$ em um ponto $(a, b)$, envolvendo a existência do limite no ponto.
    
    \item \textbf{Tipos de Descontinuidades:} Identificar diferentes tipos de descontinuidades, como descontinuidades removíveis e não removíveis.
    
%    \item \textbf{Teorema do Valor Intermediário:} Compreender o teorema do valor intermediário e sua aplicação em funções contínuas de várias variáveis.
    
%    \item \textbf{Continuidade em Conjuntos:} Explorar a continuidade em diferentes conjuntos, como conjuntos abertos, fechados e fronteiras.
    
 
\end{itemize}
}



\begin{comment}
    

Considere um economista imerso na modelagem da oferta e da demanda de um produto em um mercado complexo e em constante mudança. Nesse cenário, um conjunto de variáveis interdependentes, como o preço do produto, a renda dos consumidores, os custos de produção e outros fatores contextuais, podem afetar a quantidade demandada. À medida que nos aproximamos de uma situação específica, surge a necessidade de compreender como pequenas variações em cada um desses fatores podem alterar a quantidade de demanda - efetivamente a função em questão.

Vejamos o exemplo da relação entre o preço do produto e a demanda. Se considerarmos a função que descreve essa relação, estamos interessados em entender como uma leve elevação ou redução no preço pode influenciar a quantidade que os consumidores desejam adquirir. No entanto, essa relação é afetada por outros fatores, como a renda disponível e a existência de produtos substitutos ou complementares.

Aqui, os limites de funções de várias variáveis desempenham um papel fundamental. Eles nos permitem examinar como a quantidade demandada muda em resposta a pequenas mudanças em cada uma das variáveis independentes. Isso significa observar como a demanda se adapta quando o preço sobe ou cai, quando a renda dos consumidores aumenta ou diminui, ou quando os preços de produtos substitutos ou complementares se alteram.

Essa compreensão permite aos economistas prever as consequências de pequenas mudanças em diferentes fatores e tomar decisões informadas. Eles podem estimar como as políticas de preços, flutuações de renda ou mudanças em outros fatores influenciarão a demanda, permitindo que planejem estratégias de mercado mais eficazes.

\end{comment}


Suponha que queremos estudar a demanda por empréstimos bancários e, nesse cenário, a quantidade de empréstimos demandados ($Q$) depende da taxa de juros dos empréstimos (\( r \)) e da renda disponível (\( I \)) dos tomadores de empréstimos\footnote{Observe que em cenários reais $Q$ depende do tempo, mas por não ser uma variável contínua não a tomaremos em consideração no exemplo.}.

Queremos analisar como pequenas variações em cada uma dessas variáveis influenciam a quantidade de empréstimos demandados. Por exemplo:
\begin{enumerate}
\item \textbf{Variação na Taxa de Juros (\( r \))}: Se a taxa de juros aumentar levemente, como isso afetará a quantidade de empréstimos demandados? Podemos calcular a quantidade de empréstimos demandados \( Q \) quando \( r \) se aproxima de um valor ligeiramente maior e comparar isso com o valor original de \( Q \). Isso nos daria uma ideia de como a demanda por empréstimos responde a variações na taxa de juros.

\item \textbf{Variação na Renda Disponível (\( I \))}: Se a renda disponível das pessoas aumentar um pouco, como isso influenciará a demanda por empréstimos? Novamente, podemos calcular a quantidade de empréstimos demandados \( Q \) quando \( I \) se aproxima de um valor ligeiramente maior e comparar os resultados.

%\item \textbf{Variação no Tempo (\( t \))}: Embora o tempo não seja uma variável contínua como \( r \) ou \( I \), podemos pensar em analisar como a demanda por empréstimos evolui ao longo do tempo. Podemos observar como \( Q \) muda quando \( t \) aumenta em uma unidade (neste curso focaremos em variáveis contínuas). 
\end{enumerate}

Vejamos a seguir um exemplo muito particular.

\begin{example}{Demanda por empréstimos bancários}{exem:demanda_emprestimos}
Suponha que a quantidade de empréstimos demandados é dada pela função
\[ Q(r, I) = \frac{100}{r} \cdot I^{0.5}, \]
onde:
\begin{itemize}[label=\color{examplescolor}\textbullet]
\item \( r \) é a taxa de juros dos empréstimos;
\item \( I \) é a renda disponível dos tomadores de empréstimos. 
\end{itemize}

Vamos analisar como se comporta a função quando a taxa de juros se acerca de 5\% (ou 0.05) e a renda disponível dos tomadores de empréstimos se acerca a 3000 reais. Vejamos a seguinte matriz, que reflete os valores de \(Q(r, I)\) para alguns $(r,I)$ perto de (0.05,3000):
\[
\begin{array}{|c|c|c|c|c|c|}
\hline
 & 2600 & 2800 & 3000 & 3200 & 3400 \\
\hline
0.03 & 18257.82 & 20000.00 & 21821.89 & 23734.98 & 25757.92 \\
0.04 & 14142.14 & 15491.91 & 16970.56 & 18596.68 & 20389.09 \\
0.05 & 11547.68 & 12649.11 & \textcolor{red!140}{13856.88} & 15172.59 & 16598.76 \\
0.06 & 9718.58 & 10606.60 & 11547.68 & 12534.09 & 13559.46 \\
0.07 & 8288.62 & 9064.03 & 9898.98 & 10777.14 & 11693.75 \\
\hline
\end{array}
\]
A tabela mostra como os valores são relativamente próximos do valor de $Q(0.05,3000)$. 
\end{example}


A análise desses \textit{limites} nos permitirá entender como pequenas variações em cada uma das variáveis afetam a demanda por empréstimos. Isso é especialmente relevante para os bancos e tomadores de decisão econômica, pois ajuda a prever como as mudanças nas taxas de juros, renda disponível e tempo podem influenciar a demanda por empréstimos, permitindo o planejamento adequado das políticas financeiras e econômicas.


\begin{definition}{Limite de funções de várias variáveis}{def:limite}
Dizemos que o \textit{limite de uma função}\index{limite!de uma função}\index{função!limite de uma} $f:D\subset\R^n\to \R$ quando $\Point{x}\in D$ se aproxima de $\Point{a}\in D$ é $L$, e escrevemos
\[
\lim\limits_{{\Point{x} \to \Point{a}}} f(\Point{x}) = L,
\]
se, para qualquer $\varepsilon>0$, existe um número $\delta>0$ tal que, para todo $\Point{x}\in B(\Point{a},\delta) \cap D$, temos que 
$f(\Point{x})\in (L-\varepsilon,L+\varepsilon)\subset \R $.
\end{definition}


\begin{figure}[!htb]
    \centering
\begin{tikzpicture}[scale=1.5]

\draw[dashed,green!150,fill=green!10] (0,0) circle [radius=1];
%\draw[green!150] (30:1.5) node{\footnotesize $B(O,1)$};


\draw[-latex] (-1.5,0) -- (1.5,0) node[right]{$x$};
\draw[-latex] (0,-1.5) -- (0,1.5) node[above]{$y$};

\draw[-latex] (2,0) -- (5,0) node[right]{$z=f(x,y)$};


\draw[gray,dashed,-latex, bend left] (45:1.1) to (3.5,0.1);

\draw (2.5,0) node{\footnotesize$|$}node[below]{$L-\varepsilon$};

\draw (4.5,0) node{\footnotesize$|$}node[below]{$L+\varepsilon$};

\draw (3.5,0) node{\footnotesize$|$}node[below]{$L$};

\draw[red] (4,0) node{\footnotesize$|$} node[above]{$f(\Point{x})$};




%\draw[dash pattern= on 2pt off 2pt,red!75] (90+45:.85) circle [radius=.10] ;
%\filldraw[red!140] (90+45:.85) circle (.25pt);

\draw[dash pattern= on 2pt off 2pt,blue!140] (45:.65) circle [radius=.25];
\draw[blue!140,fill=blue!140] (45:.65) circle (.45pt) node[below]{\footnotesize $\Point{a}$};

\draw[blue!140,fill=blue!140] (45:.85) node[red,below]{\footnotesize $\Point{x}$};

\draw[blue!140] (105:.5) node[rectangle,fill=green!10,inner sep=0]{\footnotesize $B(a,\delta)$};


%\draw[dash pattern= on 2pt off 2pt,orange] (-135:.20) circle [radius=.7] ;
%\filldraw[orange] (-135:.20) circle (.45pt);



%\draw[red!150] (0,0) circle [radius=1.5];
%\draw[red!150] (45:1.5) node[above right]{\small $(x,y)$};
%\filldraw[red!150] (45:1.5) circle (1.5pt);



%\draw (0,0) circle (1.5pt) node{};
\end{tikzpicture}
\caption{}
    \label{fig:limite}
  \end{figure}

\newpage

Vamos mostrar um exemplo prático de como escolher \(\delta\) em função de \(\varepsilon\). 

\begin{example}{}{}
Vamos mostrar que 
%Queremos calcular o limite de \(f(x, y) = x^2 + y^2\) quando \((x, y)\) se aproxima de \((1, 2)\). Ou seja, estamos interessados em 
%\[\lim\limits_{{(x, y) \to (1, 2)}} f(x, y).\]
%\textbf{Afirmação:}
\[\lim\limits_{{(x, y) \to (1, 2)}} f(x, y)
%=f(1, 2) = 1^2 + 2^2 
= 5.\]
Pela definição, precisamos encontrar um número $\delta>0$ que garanta que \(|x^2+y^2 - 5| < \varepsilon\) sempre que 
$\dist((x,y),(1,2))<\delta$. 
Primeiro, temos que
\begin{align*}
    |x^2+y^2 - 5|&=|x^2+y^2 - (\sqrt{5})^2|\\
    &= |(\sqrt{x^2+y^2} - \sqrt{5})(\sqrt{x^2+y^2} + \sqrt{5})|\\
    &= |\sqrt{x^2+y^2} - \sqrt{5}||\sqrt{x^2+y^2} + \sqrt{5}|. 
\end{align*}
Da desigualdade triangular segue que 
$$\sqrt{5}=\dist((1,2),(0,0))\leq \dist((1,2),(x,y))+\dist((x,y),(0,0))<\delta + \sqrt{x^2+y^2},$$
logo, 
\begin{equation}\label{eq:mod_1}
    \sqrt{5}-\sqrt{x^2+y^2}<\delta.
\end{equation}
Analogamente, 
$$\sqrt{x^2+y^2}=\dist((x,y),(0,0))\leq \dist((x,y),(1,2))+\dist((1,2),(0,0))<\delta+\sqrt{5},$$
donde
\begin{equation}\label{eq:mod_2}
\sqrt{x^2+y^2}-\sqrt{5}<\delta.
\end{equation}
De \eqref{eq:mod_1} e \eqref{eq:mod_2} concluímos que
$$|\sqrt{x^2+y^2}-\sqrt{5}|<\delta.$$
Assim, 
\begin{align*}
    |x^2+y^2 - 5|& < \delta|\sqrt{x^2+y^2} + \sqrt{5}|. \\
    &=\delta(\sqrt{x^2+y^2} + \sqrt{5})\\
    &<\delta((\delta+\sqrt{5})+\sqrt{5})\\
    &=\delta(\delta+2\sqrt{5}).
\end{align*}
Escolhendo $\delta<\min\left\{2\sqrt{5},\frac{\varepsilon}{4\sqrt{5}}\right\}$, obtemos $|x^2+y^2 - 5|<\varepsilon$ 
como desejado. 
\end{example}
%\todo[inline]{Preciso verificar as contas!!!}

%Assim como em funções de uma variável o limite de funções de várias variáveis é único. 
\begin{theorem}{Unicidade do limite}{thm:unicidade_limite}
Se o limite de uma função de várias variáveis existe em um ponto, ele é único.   
\end{theorem}

\begin{proof}
Suponhamos por contradição que não é verdade, ou seja, que o limite de uma certa função $f:D\subset\R^n\to\R$ em um ponto $\Point{a}\in D$ existe mas que ele não é único, isto é, existem dois números reais diferentes, $L_1$ e $L_2$, tal que para todo $\varepsilon>0$ temos que: 
\begin{itemize}[label=\color{resultscolor!150}\textbullet]
\item existe $\delta_1>0$ tal que se $\Point{x}\in B(\Point{a},\delta_1)$, então $|f(\Point{x})-L_1|<\varepsilon/2$, e
\item existe $\delta_2>0$ tal que se $\Point{x}\in B(\Point{a},\delta_2)$, então $|f(\Point{x})-L_2|<\varepsilon/2$.
\end{itemize}
Sejam $\delta<\min\{\delta_1,\delta_2\}$ e $\Point{x}\in B(\Point{a},\delta)$, então
$$|L_1-L_2|=|L_1-f(x)+f(x)-L_2|\leq |L_1-f(x)|+|f(x)-L_2| < \varepsilon/2 + \varepsilon/2=\varepsilon.$$
Como $\varepsilon$ é qualquer número positivo, passamos ao limite quando $\varepsilon\rightarrow 0$ obtendo $|L_1-L_2|=0$, o que é uma contradição. Concluímos que o limite é, de fato, único. 
\end{proof}

Esse teorema é de extrema importância. Ele garante que, se o limite de uma função existe, então existe o limite da função para qualquer direção que se aproxime desse ponto. Mais ainda, temos o seguinte resultado fundamental.  
\begin{corollary}{}{}
    Se o limite de uma função de várias variáveis existe em um ponto, então existe o limite ao longo de qualquer caminho \textit{contínuo} que se aproxima desse ponto, e coincide com o limite da função no ponto.  
\end{corollary}



Vejamos um exemplo da sua aplicação. 

\begin{example}{}{}
Vamos mostrar que não existe 
%Como poderiamos Mostre que não existe
%$$\lim\limits_{(x,y)\rightarrow(0,0)}\dfrac{x^2-y^2}{x^2+y^2}.$$
\begin{equation}\label{eq:exemplo_limite}
\lim_{(x,y)\rightarrow(0,0)} \dfrac{xy\cos y}{3x^2+y^2}.
\end{equation} 
Pelo corolário anterior%\ref{thm:unicidade_limite}
, basta achar dois caminhos para os quais o limite da função 
$$f(x,y)=\dfrac{xy\cos y}{3x^2+y^2}$$ 
ao longo deles seja diferente. Podemos escolher nos aproximarmos da origem primeiramente ao longo do eixo $x$, que é o caminho
$\{(x,0)\in\R^2\}.$ 
Neste caso temos, 
$$ \lim_{(x,0)\rightarrow(0,0)} f(x,0)=\lim_{x\rightarrow 0} \dfrac{x\cdot 0 \cdot \cos 0}{3x^2+0^2}=0. $$

Agora \textit{escolhemos} acercarmos ao longo do eixo $y$, que é o caminho 
$\{(0,y)\in\R^2\}.$ 
Temos
$$ \lim_{(0,y)\rightarrow(0,0)} f(0,y)=\lim_{y\rightarrow 0} \dfrac{0\cdot y \cdot \cos y}{3\cdot 0^2+y^2}=0. $$

Embora esses dois limites sejam iguais, isso não diz nada sobre a existência do limite da função em si, pois podem existir outros caminhos para os quais o limite da função restrito a eles sejam diferentes. De fato, se nos acercamos da origem ao longo da reta de equação $y=x$, por exemplo, que é  conjunto
$\{(x,x)\in\R^2\},$ 
temos
$$\lim_{(x,x)\rightarrow(0,0)} f(x,x)=\lim_{x\rightarrow 0} \dfrac{x\cdot x \cdot \cos x}{3\cdot x^2+x^2}=\lim_{x\rightarrow 0} \dfrac{x^2 \cos x}{4x^2}= \frac{1}{4},$$
que é diferente do limite da função ao longo dos eixos coordenados (ou seja, diferente de 0). Logo, não existe o limite \eqref{eq:exemplo_limite} como queríamos mostrar. 
\end{example}



Outras propriedades importantes, cuja prova deixamos como exercício, estão listadas a seguir.
\begin{properties}{}{}
Sejam $f$, $g$ e $h$ funções que estão definidas sobre um certo domínio $D\subset\R^n$. Temos:
\begin{itemize}[label=\color{chapterscolor}\textbullet,itemsep=10pt]
\item \textbf{Soma, Produto e Quociente:} Se     
    $\lim\limits_{\Point{x}\to\Point{a}} f(\Point{x})= L_1$ e $\lim\limits_{\Point{x}\to\Point{a}} g(\Point{x})= L_2$, então:
    \begin{itemize}[label=\color{chapterscolor}\textbf{--},itemsep=10pt]
        \item $\lim\limits_{\Point{x}\to\Point{a}} [f(\Point{x}) + g(\Point{x})] = L_1 + L_2$
        \item $\lim\limits_{\Point{x}\to\Point{a}} [f(\Point{x}) \cdot g(\Point{x})] = L_1 \cdot L_2$
        \item $\lim\limits_{\Point{x}\to\Point{a}} \frac{f(\Point{x})}{g(\Point{x})} = \frac{L_1}{L_2}$ desde que $L_2 \neq 0$. 
    \end{itemize}
    
\item \textbf{Limite de uma Composição:} Se $f(\Point{x})$ tem limite $L$ quando $\Point{x}$ se aproxima de $\Point{a}$, e $g$ tem limite $l$ quando $t$ se aproxima de $L$, então:
\[\lim\limits_{\Point{x}\to\Point{a}} g(f(\Point{x})) = \lim\limits_{t\to l} g(t) = L. \]
    
%\item \textbf{Limite de uma Função Contínua:} Se $f(\Point{x})$ é contínua em $\Point{a}$, então o limite de $f(\Point{x})$ quando $\Point{x}$ se aproxima de $\Point{a}$ é igual a $f(\Point{a})$:
%\[\lim\limits_{\Point{x}\to\Point{a}} f(\Point{x}) = f(\Point{a})\]
    
\item \textbf{Propriedade do Sanduíche:} Se $h(\Point{x}) \leq f(\Point{x}) \leq g(\Point{x})$ para todos os $\Point{x}\in B(\Point{a},\delta)$, exceto possivelmente em $\Point{a}$ em si, e se $\lim\limits_{\Point{x}\to\Point{a}} h(\Point{x}) = \lim\limits_{\Point{x}\to\Point{a}} g(\Point{x}) = L$, então $\lim\limits_{\Point{x}\to\Point{a}} f(\Point{x}) = L$.
    
%\item \textbf{Limite Finito de uma Função Contínua:} Se $f(\Point{x})$ é contínua em $\Point{a}$ e $\lim\limits_{\Point{x}\to\Point{a}} f(\Point{x}) = L$, então $f(\Point{a}) = L$.
\end{itemize}

\end{properties}


\begin{comment}

\section{Sequências}

Sequências em \(\mathbb{R}^n\) são sequências de vetores que pertencem ao espaço \(\mathbb{R}^n\), onde \(n\) representa o número de componentes do vetor. Uma sequência em \(\mathbb{R}^n\) pode ser expressa como:

\[
\Point{a}_1 = (a_{11}, a_{12}, \ldots, a_{1n}),
\]
\[
\Point{a}_2 = (a_{21}, a_{22}, \ldots, a_{2n}),
\]
\[
\vdots
\]
\[
\Point{a}_k = (a_{k1}, a_{k2}, \ldots, a_{kn}),
\]
\[
\vdots
\]
onde \(\Point{a}_k\) é o \(k\)-ésimo vetor da sequência e \(a_{ki}\) é o \(i\)-ésimo componente do \(k\)-ésimo vetor.

Escrevemos $(\Point{a}_k)_{k\in\mathbb{N}}$ para denotar a sequência, mas o conjunto cujos elementos são os pontos da sequência se escreve como
$$\{\Point{a}_k,~k\in\R\}.$$
Tal conjunto é um subconjunto de $\R^n$.



A seguir mostramos o termo geral de algumas sequências em $\R^2$: 
$$
\Point{a}_k=\left(\frac{1}{k},0\right), ~ \Point{b}_k=\left(0,\frac{1}{k}\right)~\mbox{ e }~
\Point{c}_k=\left(\frac{1}{k},\frac{1}{k}\right). 
$$

Assim como em sequências de números reais, as sequências em \(\mathbb{R}^n\) também podem ter limites. Uma sequência de vetores \(\{\Point{a}_k\}\) \textit{converge} para um ponto  \(\Point{a}\) em \(\mathbb{R}^n\) se, para qualquer \(\varepsilon > 0\), existe um \(k_0\in\mathbb{N}\) tal que, para todo \(k > k_0\), a distância entre \(\Point{a}_k\) e \(\Point{a}\) seja menor que \(\varepsilon\), ou seja:
\[
\|\Point{a}_k - \Point{a}\| < \varepsilon, ~\forall~k>k_0. 
\]

Isso significa que à medida que a sequência avança, seus elementos se aproximam cada vez mais do ponto limite \(\Point{a}\).

A definição e propriedades das sequências em \(\mathbb{R}^n\) são generalizações naturais das definições correspondentes em \(\mathbb{R}\), mas agora tratando de pontos em um espaço de maior dimensão.

\begin{exercise}{avançado}{}
    Seja $(a_k)_{k\in\mathbb{N}}$ uma sequência em $\R^n$. Seja $A=\{a_k,~k\in\R\}$. 
\begin{enumerate}[label=\color{examplescolor}\textbf{(\alph*)}]
    \item Determine se $A$ é aberto, fechado ou nenhuma dessas opções.
    \item Determine $A^\circ$, $\partial A$ e $\overline{A}$. 
    \item Determine sob qué condições $A$ é limitado.
    \item Explique sobre a compacidade de $A$. 
\end{enumerate}    
\end{exercise}


\subsection{Limite de funções de várias variáveis usando sequências}

Deixamos como exercício provar que 
$$\lim\limits_{\Point{x}\rightarrow \Point{a}} f(\Point{x})=L$$
equivale a dizer que para qualquer sequência \( \{\Point{x}_k\} \) de vetores diferentes de \( \Point{a} \) que converge para \( \Point{a} \), a sequência de números reais correspondente \( \{f(\Point{x}_k)\} \) converge para \( L \). Em outras palavras: 


\begin{theorem}{}{}
Dada uma função $f:D\subset\R^n\to\R$, temos que 
$$\lim\limits_{\Point{x}\rightarrow \Point{a}} f(\Point{x})=L$$
se, e somente se, para cada sequência \( \{\Point{x}_k\} \) tal que \( \lim\limits_{{k \to \infty}} \Point{x}_k = \Point{a} \), deve-se ter \( \lim\limits_{{k \to \infty}} f(\Point{x}_k) = L \).
\end{theorem}

Essa definição é uma extensão natural da definição de limite para funções de uma variável, mas agora aplicada a funções de várias variáveis e considerando a convergência de sequências de vetores.


\begin{example}{}{}
    Mostre, usando sequências, que não existe 
    $$\lim\limits_{\Point{x}\rightarrow\Point{a}} \dfrac{xy^2}{x^2+y^4}.$$
\end{example}
\end{comment}



\section{Continuidade}


\begin{definition}{Função contínua}{}
Dizemos que uma função \( f:D\subset\R^n\to\R \) é \textit{contínua no ponto}\index{função!contínua em um ponto}  \( \Point{a}\in D \) se existe o limite quando a função se aproxima de $\Point{a}$ e esse limite é igual a $f(\Point{a})$.
\end{definition}

Resumindo, $f$ é contínua em $\Point{a}$ se 
$$\lim_{\Point{x}\rightarrow\Point{a}}f(\Point{x})=f(\Point{a}).$$
Caso $f$ seja contínua em todo ponto de $D$ dizemos que $f$ é \textit{contínua em $D$}\index{função!contínua}. 



\begin{remark}{}{}
Observe que a definição anterior é basicamente uma reescrita do conceito de limite, pois que exista tal limite e que o mesmo seja igual a $f(\Point{a})$, é o mesmo que dizer que \textit{para cada \(\varepsilon > 0\), existe \(\delta > 0\) tal que para todo \(\Point{x} \in B(\Point{a},\delta)\cap D\) temos que %$|f(\Point{x})-f(\Point{a})| < \varepsilon$
$f(\Point{x})\in (f(\Point{a})-\varepsilon, f(\Point{a})+\varepsilon)$.}
Isso significa que se fizermos uma pequena perturbação no ponto, o valor da função diferirá muito pouco do valor do ponto em questão, como ilustrado no Exemplo %\ref{exem:demanda_emprestimos}
inicial.

\end{remark}





